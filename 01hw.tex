\documentclass[12pt]{article}
\usepackage{fullpage}
\usepackage{minted}
\usepackage{enumitem}
\title{Data Structures Problem Set \#1}
\author{Nicholas Yang}
\date{18 September 2017}
\begin{document}
\maketitle
\section{Exercise 1}
\begin{enumerate}[label=(\alph*)]
\item Rectangle rotate methods:
  \begin{minted}{Java}
     public Rectangle DestRotate() {
        double temp;
        temp = xSpan;
        xSpan = ySpan;
        ySpan = temp;
        return this;
    }

    public Rectangle NonDestRotate() {
        Rectangle newRect = new Rectangle(xSpan, ySpan);
        newRect.DestRotate();
        return newRect;
    }
  \end{minted}
  LocatedRect rotate methods:
  \begin{minted}{Java}
    public LocatedRect DestRotate() {
        // First change the lengths of the sides:
        super.DestRotate();
        // Then shift it over by ySpan
        this.translateDest(super.getYSpan(), 0);
        return this;
    }

    public LocatedRect NonDestRotate() {
        LocatedRect newLocatedRect = new LocatedRect(
          xL,
          xL + xSpan,
          yL,
          yL + ySpan
        );
        return newLocatedRect.DestRotate();
    }
  \end{minted}

\item TestRotate driver program:
  \begin{minted}{Java}
    public class TestRotate {
        public static void main(String[] args) {
            LocatedRect lr = new LocatedRect(2, 10, 4, 7);
            System.out.println("Original Located Rectangle:");
            System.out.println(lr);
            System.out.println("Non destructive rotate");
            System.out.println(lr.NonDestRotate());
            // Note how the object state isn't changed
            System.out.println(lr);
            System.out.println("Destructive rotate:");
            System.out.println(lr.DestRotate());
            // Now the state has been mutated
            System.out.println(lr);
            Rectangle r = new Rectangle(4, 7);
            System.out.println("Original Rectangle:");
            System.out.println(r);
            System.out.println("Non destructive rotate");
            System.out.println(r.NonDestRotate());
            // Again, not mutated here
            System.out.println(r);
            System.out.println("Destructive rotate:");
            System.out.println(r.DestRotate());
            // But mutated here
            System.out.println(r);
        }
    }
  \end{minted}
\end{enumerate}
\section{Exercise 2}
\begin{enumerate}[label=(\alph*)]
\item Square class:
  \begin{minted}{Java}
   public class Square extends Rectangle {
       public Square(double side) {
           super(side, side);
       }
       // Doesn't matter what side
       public double getSide() {
           return super.getYSpan();
       }
       public void setSide(double side) {
           super.setSpans(side, side);
       }
       public void setSpans(double x, double y) {
           System.out.println(
               "You may not use setSpans to set the sides of a square!"
           );
       }
   }
  \end{minted}
\item LocatedSquare class:
  \begin{minted}{Java}
    public class LocatedSquare extends Square {
        // Coords of lower left corner
        private double x, y;
        public LocatedSquare(double side, double x, double y) {
            super(side);
            setCorner(x, y);
        }
        public void setCorner(double newX, double newY) {
            x = newX;
            y = newY;
        }
        // Returns rightmost x value (x2)
        public double right() {
            return x + super.getSide();
        }
        // Returns leftmost x value (x1)
        public double left() {
            return x;
        }
        // Returns highest y value (y2)
        public double top() {
            return y + super.getSide();
        }
        // Returns lowest y value (y1)
        public double bottom() {
            return y;
        }
        public String toString() {
            return "LS[ x = " + x + " y = " + y + " side = " + getSide() + "]";
        }
    }
    \end{minted}
  \item Can you do part (b) by having LocatedSquare extend both
Square and LocatedRectangle? \\
Not with traditional inheritance. You could define Square and LocatedRectangle
as interfaces, then simply have LocatedSquare implement both. However, it's
probably best to just utilize object composition with a Point class for the
location and a Square class for the actual object.
\end{enumerate}
\section{Exercise 3}
Write the following code:
\begin{enumerate}[label=(\alph*)]
\item A data field spouse, of class Person:
  \begin{minted}{Java}
    private Person spouse;
  \end{minted}
\item A getter {\tt getSpouse()}
  \begin{minted}{Java}
   public Person getSpouse() {
       return spouse;
   }
  \end{minted}
\item A method {\tt marry(Person q)}
  \begin{minted}{Java}
    // Ideally would do in intermediary class, maybe a Priest class
    public void marry(Person potentialPartner) {
        // Check if related
        boolean isRelated = potentialPartner.getParent2() == this ||
                potentialPartner.getParent1() == this ||
                this.parent1 == potentialPartner ||
                this.parent2 == potentialPartner;

        if (potentialPartner == null) {
            System.out.println("You can't marry imaginary people, " + name);
            return;
        }
        if (potentialPartner == this) {
            System.out.println("You can't marry yourself");
            return;
        }
        if (potentialPartner.getSpouse() != null) {
            System.out.println("Uh oh, they're already married!");
            return;
        }
        if (this.spouse != null) {
            System.out.println("Uh oh, you're already married!");
            return;
        }
        if (isRelated) {
            System.out.println("Ewww");
            return;
        }
        potentialPartner.setSpouse(this);
        setSpouse(potentialPartner);
    }
  \end{minted}
\item A method {\tt divorce() }
  \begin{minted}{Java}
    public void divorce() {
        if (this.spouse != null) {
            this.spouse.setSpouse(null);
            this.spouse = null;
        }
    }
  \end{minted}
\item D.i requires you to check that {\tt q} is not {\tt null}. Explain why
  there is no point in checking that {\tt p}, the owner of that method, is not null. \\

  If {\tt p} was null, there would be a NullPointerException. Plus where would
  that checking happen? {\tt p} is null, so it can't happen there. It would have
  to happen in the caller. Or just use Kotlin.

\end{enumerate}
\section{Exercise 4}
\begin{enumerate}[label=(\alph*)]
\end{enumerate}
\end{document}
