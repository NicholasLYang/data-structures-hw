\documentclass[12pt]{article}
\usepackage{fullpage}
\usepackage{minted}
\usepackage{enumitem}
\title{Data Structures Problem Set \#1}
\author{Nicholas Yang}
\date{18 September 2017}
\begin{document}
\maketitle
\section{Exercise 1}
\begin{enumerate}[label=(\alph*)]
\item For both {\tt Rectangle } and {\tt LocatedRect}, write two methods that rotate the rectangle by $90^o$
  clockwise around its lower right hand corner (see picture) \\

  Rectangle rotate methods:
  \begin{minted}{Java}
    public Rectangle DestRotate() {
      double temp;
      temp = xSpan;
      xSpan = ySpan;
      ySpan = temp;
      return this;
    }

    public Rectangle NonDestRotate() {
      Rectangle newRect = new Rectangle(ySpan, xSpan);
      return newRect;
    }
  \end{minted}
  LocatedRect rotate methods:
  \begin{minted}{Java}
    public LocatedRect DestRotate() {
      // First change the lengths of the sides:
      super.DestRotate();
      // Then shift it over by ySpan
      // (rotating around lower right corner
      //  but positioning is done by lower left corner)
      this.translateDest(super.getYSpan(), 0);
      return this;
    }

    public LocatedRect NonDestRotate() {
      LocatedRect newLocatedRect = new LocatedRect(
      xL,
      xL + xSpan,
      yL,
      yL + ySpan
      );
      return newLocatedRect.DestRotate();
    }
  \end{minted}

\item TestRotate driver program:
  \begin{minted}{Java}
    public class TestRotate {
      public static void main(String[] args) {
        LocatedRect lr = new LocatedRect(2, 10, 4, 7);
        System.out.println("Original Located Rectangle:");
        System.out.println(lr);
        System.out.println("Non destructive rotate");
        System.out.println(lr.NonDestRotate());
        // Note how the object state isn't changed
        System.out.println(lr);
        System.out.println("Destructive rotate:");
        System.out.println(lr.DestRotate());
        // Now the state has been mutated
        System.out.println(lr);
        Rectangle r = new Rectangle(4, 7);
        System.out.println("Original Rectangle:");
        System.out.println(r);
        System.out.println("Non destructive rotate");
        System.out.println(r.NonDestRotate());
        // Again, not mutated here
        System.out.println(r);
        System.out.println("Destructive rotate:");
        System.out.println(r.DestRotate());
        // But mutated here
        System.out.println(r);
      }
    }
  \end{minted}
\end{enumerate}
\section{Exercise 2}
\begin{enumerate}[label=(\alph*)]
\item Square class:
  \begin{minted}{Java}
    public class Square extends Rectangle {
      public Square(double side) {
        super(side, side);
      }
      // Doesn't matter what side
      public double getSide() {
        return super.getYSpan();
      }
      public void setSide(double side) {
        super.setSpans(side, side);
      }
      public void setSpans(double x, double y) {
        System.out.println(
        "You may not use setSpans to set the sides of a square!"
        );
      }
    }
  \end{minted}
\item LocatedSquare class:
  \begin{minted}{Java}
    public class LocatedSquare extends Square {
      // Coords of lower left corner
      private double x, y;
      public LocatedSquare(double side, double x, double y) {
        super(side);
        setCorner(x, y);
      }
      public void setCorner(double newX, double newY) {
        x = newX;
        y = newY;
      }
      // Returns rightmost x value (x2)
      public double right() {
        return x + super.getSide();
      }
      // Returns leftmost x value (x1)
      public double left() {
        return x;
      }
      // Returns highest y value (y2)
      public double top() {
        return y + super.getSide();
      }
      // Returns lowest y value (y1)
      public double bottom() {
        return y;
      }
      public String toString() {
        return "LS[ x = " + x + " y = " + y + " side = " + getSide() + "]";
      }
    }
  \end{minted}
\item Can you do part (b) by having LocatedSquare extend both
  Square and LocatedRectangle? \\
  Not with traditional inheritance. You could define Square and LocatedRectangle
  as interfaces, then simply have LocatedSquare implement both. However, it's
  probably best to just utilize object composition with a Point class for the
  location and a Square class for the actual object.
\end{enumerate}
\section{Exercise 3}
Write the following code:
\begin{enumerate}[label=(\alph*)]
\item A data field spouse, of class Person:
  \begin{minted}{Java}
    private Person spouse;
  \end{minted}
\item A getter {\tt getSpouse()}
  \begin{minted}{Java}
    public Person getSpouse() {
      return spouse;
    }
  \end{minted}
\item A method {\tt marry(Person q)}
  \begin{minted}{Java}
    // Ideally would do in intermediary class, maybe a Priest class
    public void marry(Person potentialPartner) {
      // Check if related
      boolean isRelated = potentialPartner.getParent2() == this ||
      potentialPartner.getParent1() == this ||
      this.parent1 == potentialPartner ||
      this.parent2 == potentialPartner;

      if (potentialPartner == null) {
        System.out.println("You can't marry imaginary people, " + name);
        return;
      }
      if (potentialPartner == this) {
        System.out.println("You can't marry yourself");
        return;
      }
      if (potentialPartner.getSpouse() != null) {
        System.out.println("Uh oh, they're already married!");
        return;
      }
      if (this.spouse != null) {
        System.out.println("Uh oh, you're already married!");
        return;
      }
      if (isRelated) {
        System.out.println("Ewww");
        return;
      }
      potentialPartner.setSpouse(this);
      setSpouse(potentialPartner);
    }
  \end{minted}
\item A method {\tt divorce() }
  \begin{minted}{Java}
    public void divorce() {
      if (this.spouse != null) {
        this.spouse.setSpouse(null);
        this.spouse = null;
      }
    }
  \end{minted}
\item D.i requires you to check that {\tt q} is not {\tt null}. Explain why
  there is no point in checking that {\tt p}, the owner of that method, is not null. \\

  If {\tt p} was null, there would be a NullPointerException. Plus where would
  that checking happen? {\tt p} is null, so it can't happen there. It would have
  to happen in the caller.

\end{enumerate}
\section{Exercise 4}
Consider the code for Hwk1Ex4A.java and Hwk1Ex4B.java on the attached handout \\
\begin{enumerate}[label=(\alph*)]
\item What do these output? \\
  Hwk1Ex4A.java outputs:

  {\tt
    p = 1 \\
    q = 184 \\
    r = 184
  }

  Hwk1Ex4B.java outputs:

  {\tt
    p = 175 \\
    q = 45 \\
    r = 25 \\
    s = 175
  }

\item Explain these output \\
  \begin{enumerate}

  \item   Hwk1Ex4A.java defines two classes: A and B. A has two instance variables: name
    of type String and value of type int. A has a function {\tt f} that updates
    the value by adding a parameter {\tt x} and 100 to the current value. B
    extends A and replaces the {\tt f} function with a function {\tt f} that
    updates value by adding a parameter {\tt x} and 20 to the current value.

    The main program starts by creating {\tt p} with class {\tt A} and value 1 and
    {\tt q} with class {\tt B} and value 1. From there, it sets {\tt r} with class
    {\tt A} to {\tt q}. Then, by setting {\tt r}'s value to 10, {\tt q}'s value is
    also set to 10. Afterwards it calls {\tt q.f} with {\tt p.value} as an
    argument. This sets both {\tt q} (and {\tt r}'s) value to $10 + 1 + 20$ or 31.
    Calling {\tt r.f } on {\tt q.value} also changes this value, instead this time
    to $ 31 + 31 + 20 $ or 82. Finally, we repeat this exact same process, only
    with {\tt q} and {\tt r} switched (they reference the same object so this does
    the exact same thing as above). This gives us a total of $82 + 82 + 20$ or
    $184$ for q and r's value. Since {\tt p}'s value has never been changed, this
    gives us a final output of 1, 184, and 184. Note that since {\tt q/r} were
    referencing an object of type {\tt B}, Java dynamically dispatches the {\tt f}
    method for {\tt B}

  \item Hwk1Ex4B.java defines two classes: {\tt A } and {\tt B}. {\tt Hwk1Ex4B} also
    defines two static functions, both named {\tt f}. One takes an object of class
    {\tt A} and one takes an object of class {\tt B}. The first one mutably adds
    100 to the object's value, while the second immutably adds 20 to the object's
    value.

    The main program initializes three variables, {\tt p} with declared and
    actual type {\tt A} and value 1, {\tt q} with declared and actual type {\tt
      B} and value 5 and {\tt r} with declared and actual type {\tt B} and value
    22. Then, {\tt r} is assigned to {\tt f(q)}. Since {\tt q} has a declared
    type of {\tt B}, {\tt f(q)} resolves to the {\tt B} function. Therefore {\tt
      r} now has a value of $ 5 + 20 = 25$. From there, {\tt q} is assigned the
    output of {\tt f(r)}. Since {\tt r} has a declared type of {\tt B}, {\tt
      f(r)} resolves to the {\tt B} function. This gives {\tt q} a value of $25
    + 20 = 45$. Next, a new variable {\tt s}, with declared type {\tt A} gets
    assigned to {\tt f(r)}. Therefore, {\tt s} has a actual type {\tt B} and a
    value of 45. After that, {\tt p} is assigned to {\tt f(s)}. Since {\tt s}
    has a declared type of {\tt A}, {\tt f(s) } resolves to the {\tt A}
    function. Therefore, {\tt s}'s value is incremented by 100, to become 145.
    Additionally, {\tt p} now references the same object as {\tt s}, so its
    value is also 145. Finally, {\tt p}'s value is incremented by 30, which also
    increments {\tt s}'s value to a final value of 175.
  \end{enumerate}
\end{enumerate}
\end{document}
